%\documentclass[a4paper,12pt]{article}
\documentclass{article}

\usepackage[french]{babel}
\usepackage[utf8]{inputenc}  
\usepackage[T1]{fontenc}
\usepackage{graphicx}
\usepackage{hyperref}
\usepackage{pdfpages}
%\usepackage{ifpdf}
%\ifpdf %-- si pdflatex
%  \usepackage[pdftex,colorlinks=true,urlcolor=blue,pdfstartview=FitH]{hyperref}
%  \DeclareGraphicsExtensions{.pdf,.pdftext,.png,.jpg}
%  \pdfcompresslevel=9
%\else %-- si latex
%  \DeclareGraphicsExtensions{.eps,.ps,.pst}
%\fi



\title{Documentation système personnelle}
\author{Nicolas Vacelet}
\date{\today}

\begin{document}
\maketitle

\begin{abstract}
    \centering Documentation système à des fins personnelles
\end{abstract}

\tableofcontents
\newpage

\section{Introduction}
\paragraph{}
Cette documentation est créée au fil de l'eau en guise de pense bête pour servir de fururs besoins.

\section{Latex}\label{latex}

\subsection{Installation}
\paragraph{}
L'installation de Latex est somme toute classique :
\begin{verbatim}
 $ sudo apt-get install texlive
 $ sudo apt-get install texlive-base texlive-binaries texlive-extra-utils \
	 texlive-font-utils texlive-fonts-recommended texlive-generic-recommended \
	 texlive-latex-base texlive-latex-extra \
	 texlive-pictures vim-latexsuite texlive-publishers
$ sudo apt-get install texlive-base texlive-binaries exlive-extra-utils \
	 texlive-font-utils texlive-fonts-recommended texlive-generic-recommended \
	 texlive-latex-base texlive-latex-extra texlive-pictures vim-latexsuite texlive-publishers
$ sudo apt-get install texlive-lang-english
$ sudo apt-get install texlive-base
$ sudo apt-get install  texlive-lang-french
$ sudo apt-get install xpdf
$ sudo apt-get install latexmk
\end{verbatim}

\subsection{Utilisation}
\paragraph{}
vim du fichier .lex d'un côté et latexmk -pvc -pdf filename.tex de l'autre pour avoir le rendu
\paragraph{}
Enfin texi2pdf doc.tex pour générer définitivement le pdf.


\section{Vim}\label{vim}
\paragraph{}
Les lignes suivantes sont ajoutées à ~/.vimrc pour plus de confort visuel
\begin{verbatim}
 $ apt-get install vim
 $ vim ~/.vimrc
   set cindent
   set hlsearch
   set ic
   set incsearch
   set number
   syntax on
   colorscheme light
\end{verbatim}

\section{Jenkins}\label{jenkins}
\paragraph{}
La config se trouve dans /var/lib/jenkins

\section{lxc}\label{lxc}
\subsection{conf réseau}
\paragraph{}

\subsection{Commandes}
\section{Jenkins}\label{jenkins}
\paragraph{}
lxc-create -n name -t debian
lxc-start -n name
lxc-stop -n name
lxc-attach -n name
sudo lxc-list
L'IP du container se trouve dans le proc de du container
\begin{verbatim}
  $ IP=$(awk -F'[ \t;]*'  '/fixed-address/ {IP=$3;} END {print IP;}' \
		 /var/lib/lxc/$VM_CONTAINER/rootfs/var/lib/dhcp/dhclient.eth0.leases)
  $ lxc-info -n $VM_CONTAINER
\end{verbatim}


\paragraph{}
Plugins à installer
\begin{itemize}
  \item Green Balls Plugin
  \item Parameterized Trigger Build Plugin 
  \item Build Pipeline Plugin
\end{itemize}

\subsection{RestAPI}
\paragraph{}
Pour lancer un job en REST API :  POST user:token@IP/job/MYJOB/buildWithParameters? \ 
   PARAMETER=Value\&PARAMETER=Value

\section{Divers}\label{divers}
\paragraph{}
Les commandes suivantes sont listées pour pense bête :
\begin{itemize}
  \item tee : écrit en sortie ce qu'il a récupéré en entrée : echo "hello" | tee out.txt > /dev/null
  \item sudo for ever : sudo bash
  \item echo \$(</etc/fstab) équivaut à un cat
  \item echo * équivaut à un ls
  \item arandr pour gérer plusieurs écrans
  \item lshw pour connaitre les détails du hardware
  \item evince pour lire les pdf en ligne de commande
\end{itemize}

\section{Documentation}\label{documentation}
\paragraph{}
\begin{itemize}
  \item http://it-ebooks.info
\end{itemize}

\end{document}
